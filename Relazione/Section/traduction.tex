\section{Traduzione}
	Presentiamo il codice per tradurre le triple di partenza nei due formati presentati. Discutiamo delle implicazioni delle due traduzioni in seguito
	\subsection{RDF/XML}
		\addCode{RDF/XML persona}{lst:info_to_pers_rdf}{ocaml}{Code/info_to_pers_rdf.cd}
		Applicato questa funzione alla prima informazione del Listato \ref{lst:raw_data} oteniamo:
		\addCode{Traduzione persona RDF}{trad_pers_rdf}{XML}{Code/trad_pers_rdf.xml}
		Con questo codice trasformiamo un elemento di tipo \texttt{Info} in un elemento di tipo \texttt{Individual},cioè passiamo da una tripla a un individuo della classe persone. Per completare la traduzione abbiamo poi bisogno di un'altra funzione per trasfomrmare i complementi in induvidui.
		
		Vediamo che in questo formalismo descriviamo l'azione che collega il soggetto al complemento nello stesso elemento XML che descrive anche la persona. Questo permette una sintassi più compatta e leggibile, ma ci costringe a conoscere a prescindere tutti i verbi che possono collegare un soggetto a un complemento (elencati da riga 7 a 13 del Listato \ref{lst:info_to_pers_rdf}), tutti i verbi non contemplati possono essere mappati in un solo verbo, perdendo però la specificità dell'azione.
		
		La ragione per cui si devono conoscere i verbi è dovuta al fatto che in questo formalismo il tipo di azione è rappresentato da tag, come abbiamo accennato questi elementi sono di tipo \texttt{Atom}, gli elementi di questo tipo rappresentano la struttura ad albero del documento XML e non possono essere creati durante l'esecuzione del programma perché non si potrebbe garantire la correttezza della trasformazione a nel momento della compilazione.
	\subsection{OWL/XML}
		\addCode{OWL/XML informazione}{lst:info_to_ont_owl}{ocaml}{Code/info_to_ont_owl.cd}
		Applicato questa funzione alla prima informazione del Listato \ref{lst:raw_data} oteniamo:
		\addCode{Traduzione informazione OWL}{trad_info_owl}{XML}{Code/trad_owl.xml}
		Come si vede in questo caso separiamo la descrizione della persona, da quella dell'azione da quella del complemento, In questo formalismo le azioni non devono essere note a priori, essendo l'azione il valore di un elemento del record e non più un tag.
		
		 Come si vede dalla traduzione questo formalismo disperde di più le informazioni e risulta di conseguenza più verboso e meno comprensibile, il vantaggio, come accennato prima è quello di poter descrivere una qualsiasi azione con precisione senza dover conoscere tutte le azioni a priori. Questo formalismo risulta più espressivo e generale; offre la possibilità di descrivere manipolazioni in modo più flessibile con \cduce continuando comunque a poter beneficiare del cotnrollo statico sui tipi offerto dal linguaggio.
		 
	\subsection{Differenze}
		Possiamo individuare la differenza principale tra i due formati nella gestione delle relazioni tra soggetto e complemento:
		\begin{itemize}
			\item \textbf{RDF}: in questo formato è il soggetto l'elemento principale della nostra descrizione, tra i vari elementi che descrivono il soggetto abbiamo i campi che riguardano le relazioni a cui il soggetto partecipa, in questo senso è il soggetto che è collegato tramite un'azione al complemento. Questo approccio è dovuto al fatto che RDF è un formalismo più di basso livello in cui le relazioni non sono la parte centrale della descrizioni degli individui
			\item \textbf{OWL}: questo formato invece pone enfasi sulle relazioni che si instaurano tra gli individui descritti nell'ontologia; soggetto e complemento sono individui separati e indipendent. L'azione che in questo formato è l'elemento principale della descrizione ed è collegata al soggetto e al complemento in modo da unirli.
		\end{itemize}