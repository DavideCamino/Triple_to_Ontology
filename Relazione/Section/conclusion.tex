\section{Conclusioni}
	\subsection{Risultati ottenuti}
	In questo lavoro siamo riusciti a arricchire un set di informazioni dando loro una struttura formale grazie alla quale la ricerca e la manipolazione sia supportada da potenti strumenti sviluppati appositamente per le ontologie. 
	
	Abbiamo ottenuto differenti formati dell'ontologia, ogniuno con vantaggi e svantaggi specifici. Il primo formato descritto è più rigido per quanto riguarda la creazione e la manipolazione, mentre il secondo ci permette più flessibilità.
	
	Dal punto di vista della complessità i formati si scambiano, il primo è più semplice ed è il più adatto ad altre manipolazioni machine oriented, può ad esempio essere l'input di progrmmi di apprendimento automatico. Il secondo più espressivo risulta più complesso e non particolarmente adatto a algoritmi di AI. 
	
	Seppure sia possibile tradurre un formato nell'altro attraverso strumenti differenti da \cduce si vorrebbe, per questioni di overhead evitare una catena continua di traduzioni per passare continuamente da un formato all'altro in base a quello più comodo per la specifia elaborazione.
	\subsection{Lavoro futuro}
		\subsubsection{Duplicati nella traduzione}
			Per quanto riguarda la traduzione attualmente si fa in modo che due triple differenti rappresentino lo stesso soggetto semplicemente duplicando la rappresentazione, gli strumenti di visualizzazione delle ontologie sono in grado di evidenziare questi duplicati e eliminarli automaticamente, sarà necessario modificare le funzioni di traduzione in modo che questi duplicati non vengano proprio creati.
		\subsection{Manipolazione}
			Mostrare, una volta fatta la traduzione come sia possibile manipolare i dati arricchiti dalla struttura dell'ontologia attraverso strumenti appositi come le query in Description Logic e come queste possano essere espresse in \cduce evidenziando le differenze tra i due formati.
		\subsubsection{Struttura dell'ontologia}
			Dato che adesso stiamo rappresentando i dati con una struttura formale possiamo aggiungere sottoclassi per raggruppare gli individui in base a particolari caratteristiche di nostro interesse, ad esempio per uno studio sui bias di rappresentazione potrebbe essere utile raggruppare le persone in caregorie per valutare se ci sono differenze sostanziali nelle azioni che coinvolgono le varie sottoclassi di persone.
	
	