\section{Ontologia di arrivo}
	Descriviamo brevemente come sarà fatta l'ontologia che vorremmo ottenere.
	\subsection{Struttura}
		L'ontologia che vogliamo creare per dare una struttura formale ai dati grezzi sarà costituita, come ogni ontologia da classi, individui e relazioni.
		\subsubsection{Classi}
			L'ontologia che vogliamo ottenere deve rappresentare 2 classi fondamentali:
			\begin{itemize}
				\item Persone: la classe che contiene i soggetti delle nostre triple, rappesenta il concetto di persona e tutte le caratteristiche che derivano dall'appartenere alla specie Homo Sapiens Sapiens;
				\item Complementi: questa classe rappresenta gli oggetti su cui sono svolte le azioni. è una classe estremamente eterogenea dato che non conterrà solamente oggetti fisici, ma anche astratti quali idee, concetti e pensieri
			\end{itemize}
		\subsubsection{Individui}
			Gli individui apparterranno a una delle due classi definite precedentemente, vogliamo fare in modo che se due triple di partenza parlano dello stesso soggetto o dello stesso oggetto questo sia rappesentato da un solo individuo anche nell'ontologia che creiamo. Ad esempio:
			\begin{itemize}
				\item Crick, Franklin e Watson scoprirono la struttura del DNA, da queste 3 affermazioni vorremmo ottenere 3 individui di classe Persona e un solo individuo di classe Complemento;
				\item Albert Einstein scoprì l'effetto fotoelettrico e ideò la teoria della relativià, da queste 2 triple vorremmo ricavare 2 complementi ma un solo soggetto.
			\end{itemize}
		\subsubsection{Relazioni}
			Per ogni tripla vogliamo aggiungere una relazione alla nostra ontologia che colleghi il soggetto al complemento mediante la psecifica azione; Ovviamente vorremmo collegare più soggetti allo stesso complemento (anche mediante azioni differenti) e collegare più complementi allo stesso soggetto.
	\subsection{Formalismo}
		Esistono diversi modi per esprimere struttura e contenuto di un'ontologia, alcuni formalismi sono più espressivi di altri e pagano questo aumento di esperessività con una complessità maggiore sia per quanto riguarda la struttura stessa dell'ontologia che per ciò che riguarda gli algoritmi che faranno inferenza e/o reasioning sui dati contenuti nell'ontologia.
		
		Come già detto nella sezione precedente il linguaggio di programmazione che usiamo è specifico per la manipolazione di documenti XML; prendiamo quindi spunto dalle ontologie create con protegé in particolare valuteremo pro e contro di documetni creati attenendosi ai formati di protegé RDF/XML e  OWL/XML