\section{Manipolazione}
	Ora che abbiamo ottenuto delle ontologie dalle informazioni partenza possiamo fare in modo da sfruttare la struttura formale per poter fare inferenza sui dati, manipolarli e cercare informazioni in modo più efficace rispetto al set di partenza.
	
	In questa sezione mostriamo come implementare una semplice query per cercare tutti i pittori nella nostra ontologia, sfruttiamo questo esempio molto semplice per confrontare 3 differenti approcci, che sfruttino l'iontologia appena creata.
	\subsection{Description Logic}
		Dato che abbiamo ottenuto un'ontologia (a prescindere che sia in formato RDF o WOWL) possiamo adesso sfruttare strumenti molto potensi sviluppati appositamente per lavorare su queste rappresentazioni formali dei concetti. In questo esempio sfruttiamo Protegé e uno dei reasoner offerti da quest'ultimo (HermiT). Grazie al reasoner possiamo interrogare l'ontologia attraverso delle query in Description Logic (DL). Per trovare tutti i pittori ad esempio possiamo semplicemente chidere:\\
		\texttt{painted \color{purple}some\color{black}}\\
		e otterremo come risposta tutti gli individui che partecipano alla relazione \say{painted}
	\subsection{\cduce \& RDF}
		Vediamo ora come sia possibile ottenere le stesse informazioni utilizzando  partendo da un'ontologia in formato RDF attraverso una funzione in \cduce
		
		Questo codice ci permette di ottenere la lista delle persone che partecipano alla relazione \say{painted}. In questo caso usiamo le query offerte da \cduce per trovare tutti gli individui che facciano il match con l'espressione descritta nella clausola where della query. Come si vede il codice è piuttosto semplice e si basa sul fatto che conosciamo esattamente il tipo del tag della relazione.
		
		In questo formato le azioni sono ristrette a quelle esattamente descritte nella funzione di trasformazione, questa rigidità nella creazione ci permette una relativa snellezza nella ricerca delle informazioni; i problemi sorgono nel caso in cui l'ontologia che vogliamo manipolare non è stata creata da noi ma ci viene assegnata, in questa situazione dobbiamo conoscere esattamente il tipo della relazione cercata per costruire una query ad hoc.
		
	\subsubsection{\cduce \& OWL}
		Se il formato dell'ontologia che vogliamo manipolare è OWL possiamo scrivere la funzione di ricerca in questo modo:
		
		